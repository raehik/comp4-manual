\section{Introduction \& features}

\todo{Is this actually detailed desc. of full sys?}

\textbf{Organisr} is a digital calendar and to-do list application which makes
tracking your appointments and daily tasks much easier. The rationale behind
Organisr is to \textit{improve and streamline} the most common organisation and
time management systems by providing a digital substitute. To this end, we
included:

\begin{itemize}
    \item a simple and easy-to-use 'to-do' list system
    \item a flexible calendar on which users can set appointments and events for
        specified days
    \item desktop notifications for appointments
    \item physical printouts of daily, weekly and monthly calendar views,
        showing any appointments on those days
\end{itemize}

Using calendars is a widespread practice, as it provides an effective way of
managing projects and events, and lets you drop certain things from your memory
so that you can concentrate on the present. An example: say you need to remember
to buy a present for a friend's birthday in the coming weeks. Holding it in
your memory for the whole two weeks would ensure that you wouldn't forget --
however many people prefer to concentrate absolutely on one task at a time.
So you'd note down their birthday on a calendar you check frequently and sort
it out when you find some spare time.

Organisr improves on the traditional calendar by allowing \textbf{time-based
notifications} to be set for individual appointments. Using this method, events
such as meetings and birthdays become a cinch to remember, since you can make a
notification for whenever you need and rest assured that you'll be reminded on
time.

For people who prefer to use 'material' time management systems (or if someone
asks for a copy of your project schedule), Organisr allows you to make
\textbf{paper printouts} for your to-dos and appointments.


\section{Installation}

\subsection{Linux}

\subsubsection{Building from source}

Ensure that the \href{http://srombauts.github.io/SQLiteCpp/}{SQLiteC++} and
\href{http://sqlite.org/download.html}{SQLite} libraries are installed.
\verb|libSQLiteCpp| and \verb|libsqlite3| should both in your your library path
(you may need to copy the compiled SQLiteC++ library file to
\verb|/usr/local/lib| manually), and all of the SQLiteC++ header files should be
in your include path. Organisr uses qmake, so best practices suggest making an
out-of-source build. First, download the source from GitHub using Git or another
method:

\begin{lstlisting}
git clone https://github.com/raehik/organisr
cd organisr
\end{lstlisting}

Now run qmake, and run \verb|make| on the resulting Makefile:

\begin{lstlisting}
mkdir build
qmake ..
make
\end{lstlisting}

\begin{verbatim}
mkdir build
qmake ..
make
\end{verbatim}


\subsection{Windows}

Run the provided installer.
