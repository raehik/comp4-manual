% vim: spell ft=tex
\section{Using Organisr}

Organisr's main two features are its \textbf{to-dos} and \textbf{appointments}.
Using these, you can track things you need to remember, and plan ahead for
meetings or events. Organisr also lets you \textbf{search} through your previous
and upcoming appointments by search term or a specific date, so you can plan
ahead for birthdays and such -- or check when that meeting you missed was!


\subsection{Using to-dos}

\smallimg{01-todo-main}

The default tab for Organisr's main window is the \guiel{To-dos} tab. This tab
lets you view your current to-dos. Using the buttons on the left of each one,
you can also tick off completed tasks, edit ones which need changing, and delete
old or finished ones.


\subsubsection{Adding to-dos}

\smallimg{02-todo-add}

On the \guiel{To-do} tab, click the \guiel{Add new to-do} button at the bottom
of the window. A dialog box will open and prompt you for the to-do details. Type
in the text you would like have to displayed on list (e.g. `Send Mary the
project email') and click \guiel{Confirm}. You can hit \guiel{Cancel} to cancel
the new to-do operation.

\tip{Keyboard shortcut for adding a to-do: Alt+A on the to-do tab.\\
You can also hit Alt+N, Alt+S, Alt+V and Alt+H to drop down the respective
toolbar button, and use the arrows keys and Enter to choose a entry.}


\subsubsection{Editing to-dos}

\smallimg{03-todo-edit}

You can also edit existing to-dos by clicking the \guiel{`E'} button to the left
of each to-do, which may be useful if you forget to write some important detail.
This will bring up a similar dialog and will change the selected to-do to match
whatever you type in the box when you hit \guiel{Confirm}. To cancel the edit,
click \guiel{Cancel}.


\subsubsection{Completing to-dos}

\halfimg{04-todo-complete-1}
\halfimg{04-todo-complete-2}

When you finish to-dos, you might like to keep a running count of completed
tasks on that day or shift, or you need to know what you were able to do during
a certain time. Organisr lets you 'complete' to-dos by clicking the
checkmark button (`✓').

Completed to-dos are moved to the bottom of the list so that you can concentrate
on pending tasks.


\subsubsection{Deleting to-dos}

\halfimg{05-todo-delete-1}
\halfimg{05-todo-delete-2}

Clicking on the \guiel{`X'} button to the left of any to-do will prompt you to
delete it. Deleted to-dos \textbf{cannot be recovered}, so be careful!

\tip{If you have finished a task but you think you might need to refer back to
it later, it might be better to mark the to-do as \textit{completed} rather than
deleting it.}


\subsubsection{Printing a paper to-do list}

\halfimg{06-todo-print-1}
\halfimg{06-todo-print-2}

If you'd like to have a paper copy of your pending to-dos, click on the
\guiel{Print} button found below the last to-do. You can send it to a printer
you're connected to, or save it to a PDF file to print later.

\tip{Note that this only prints your \textit{pending/incomplete to-dos}. So
don't worry about deleting all the crossed-out ones before printing!}


\subsection{Using appointments}

\fullimg{01-appt-main}

The simplicity of to-dos makes them very useful for taking down quick notes or
keeping track of tasks in a project. If you need to take down a bit more
information about an event, you're probably looking for an \textbf{appointment}.
Clicking on the \guiel{Appointments} tab on Organisr's main window will bring
you to the main appointments view. From here, you can view your appointments
today and on other days, edit appointments and add new ones, and make a printout
of the appointments on a single day or month.


\subsubsection{Creating an appointment}

\fullimg{02-appt-create}

On the \guiel{Appointment} tab, click the \guiel{Add new appointment} button at
the bottom of the window. A dialog box will open and prompt you for some
details. You need to at least enter a title, a date and a time. The
\textit{title} should be a concise summary of the event; if you need more
detail, add it to the \textit{description}. If you want to take a note of where
the appointment will take place, put it in the \textit{location}. When you're
happy, click \guiel{Confirm} to add the appointment to the day you selected.

\tip{Keyboard shortcut for creating an appointment: Alt+A on the appointment
tab.}


\subsubsection{Editing appointments}

\fullimg{03-appt-edit}

Just the same as with to-dos, you are able to amend any appointment that you
would like to change by clicking the \guiel{`E'} button next to it.


\subsubsection{Deleting appointments}

\fullimg{04-appt-delete}


\subsubsection{Searching for specific appointments}
