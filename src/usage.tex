\section{Using Organisr}

Organisr's main two features are its \textbf{to-dos} and \textbf{appointments}.
Using these, you can track things you need to remember, and plan ahead for
meetings or events. Organisr also lets you \textbf{search} through your previous
and upcoming appointments by search term or a specific date, so you can plan
ahead for birthdays and such -- or check when that meeting you missed was!


\subsection{Using to-dos}

\newcommand{\smallimg}[1]{\includegraphics[width=0.6\textwidth]{#1}}
\newcommand{\halfimg}[1]{\includegraphics[width=0.5\textwidth]{#1}}

\smallimg{01-todo-main}

The default tab for Organisr's main window is the \guiel{To-dos} tab. This tab
lets you view your current to-dos. Using the buttons on the left of each one,
you can also tick off completed tasks, edit ones which need changing, and delete
old or finished ones.


\subsubsection{Adding to-dos}

\smallimg{02-todo-add}

On the \guiel{To-do} tab, click the \guiel{Add new to-do} button at the bottom
of the window. A dialog box will open and prompt you for the to-do details. Type
in the text you would like have to displayed on list (e.g. `Send Mary the
project email') and click \guiel{Confirm}. You can hit \guiel{Cancel} to cancel
the new to-do operation.


\subsubsection{Editing to-dos}

\smallimg{03-todo-edit}

You can also edit existing to-dos by clicking the \guiel{`E'} button to the left
of each to-do, which may be useful if you forget to write some important detail.
This will bring up a similar dialog and will change the selected to-do to match
whatever you type in the box when you hit \guiel{Confirm}. To cancel the edit,
click \guiel{Cancel}.


\subsubsection{Completing to-dos}

\halfimg{04-todo-complete-1}
\halfimg{04-todo-complete-2}

When you finish to-dos, you might like to keep a running count of completed
tasks on that day or shift, or you need to know what you were able to do during
a certain time. Organisr lets you 'complete' to-dos by clicking the
checkmark button (`✓').

Completed to-dos are moved to the bottom of the list so that you can concentrate
on pending tasks.


\subsubsection{Deleting to-dos}

\halfimg{05-todo-delete-1}
\halfimg{05-todo-delete-2}

Clicking on the \guiel{`X'} button to the left of any to-do will prompt you to
delete it. Deleted to-dos \textbf{cannot be recovered}, so be careful!

\tip{If you have finished a task but you think you might need to refer back to
it later, it might be better to mark the to-do as \textit{completed} rather than
deleting it.}


\subsection{Using appointments}

Clicking on the \guiel{Appointments} tab on Organisr's main window will bring
you to the \textbf{main appointments view}. From here, you can view your
appointments today and on other days, edit existing appointments and add new
ones.
